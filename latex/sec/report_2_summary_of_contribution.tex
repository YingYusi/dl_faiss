\section{Summary of contribution}
While reading the REPLUG~\cite{REPLUG} paper, we found that the study presents an effective approach to determining the similarity between two texts by combining a retrieval-augmented framework with a language model, which significantly improves the detection of semantic similarity. However, the main focus of REPLUG is on overall similarity evaluation rather than precisely locating specific parts of the text that exhibit plagiarism. This results in certain limitations in its practical application, especially in educational contexts, where teachers and evaluators not only need to know whether plagiarism exists but also need to clearly identify the specific plagiarized sections in order to more accurately assess the nature and severity of the plagiarism.

Additionally, due to potential errors in artificial intelligence applications, such as mistakenly identifying unrelated content as plagiarism or overlooking potential plagiarized sections, relying solely on the model's output may not fully meet the needs of plagiarism detection. Therefore, we propose an improved solution: by leveraging the powerful language understanding capabilities of large language models, we aim to quickly pinpoint the specific sections of text where potential plagiarism occurs. This approach provides an initial screening that offers reference and analysis for human evaluators, enabling them to focus more on key issues and perform more detailed and accurate assessments within a limited timeframe.