\section{Discussion}

In this experiment, our main goal was to distinguish between three types of cases: non-plagiarism, explicit plagiarism, and implicit plagiarism. However, during the implementation process, our experimental approach had certain limitations and shortcomings. First, using cosine similarity as the criterion for similarity detection, although it can measure the similarity between texts to some extent, also has obvious flaws. The calculation of cosine similarity is based on the vector space model, which primarily focuses on the directional similarity of text vectors while ignoring the specific positional relationships of words and phrases in the text. Therefore, for implicit plagiarism—especially in cases where the sentence order is rearranged to avoid detection—cosine similarity performs poorly. In some cases, even if the sentence order changes, as long as the arrangement of words and the overall meaning don't change significantly, cosine similarity may fail to effectively detect plagiarism.

In addition, our threshold for similarity had some arbitrary aspects. We used a fixed threshold of 0.5 to judge whether plagiarism occurred, but this value may not be optimal. It is possible that a threshold other than 0.5 could more accurately reflect the criteria for detecting plagiarism, but we did not conduct in-depth research or experimental verification on this matter. If we had tested and optimized various thresholds systematically, we might have achieved more convincing results in identifying plagiarism.

Overall, due to time constraints, the design and implementation of this experiment were rushed, leading to many shortcomings. For example, we did not conduct thorough experimental exploration in areas such as text preprocessing, model selection, and parameter adjustment, nor did we compare the effectiveness of different methods. Additionally, the scale and diversity of the experimental samples were not fully considered, which may have caused biases in the results and incomplete conclusions. Therefore, for future research in similar areas, we will need more time and resources to conduct systematic optimization and verification in order to improve the accuracy and robustness of the experimental methods.