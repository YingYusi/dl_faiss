\section{Introduction}
With the rapid development of large language models (LLMs) and retrieval tools in the field of natural language processing, their applications have gradually permeated the education and academic sectors, especially showing great potential in areas such as automated grading, personalized learning, and content generation. However, with the widespread application of these technologies, academic integrity issues have become increasingly complex and severe. Problems like assignment plagiarism, ghostwriting, and code plagiarism not only undermine the fairness of the academic environment but also present significant challenges for educational institutions in terms of management. Traditional plagiarism detection methods mainly rely on simple text matching or manual inspection, which, while effective in identifying explicit textual similarities, often fail to detect implicit plagiarism or modified plagiarized texts.

Currently, the academic community has begun to explore more advanced plagiarism detection methods, including semantic matching techniques based on large language models and retrieval-augmented frameworks. These methods offer deeper text understanding and can provide support in more complex plagiarism scenarios. However, most existing studies focus on determining the overall similarity between texts, and there is still a lack of effective solutions for precisely pinpointing specific plagiarized sections of the text. Therefore, in this paper, we propose an improved approach based on adjusting the parameters of FAISS~\cite{FAISS} and REPLUG~\cite{REPLUG}, aiming to enhance the accuracy and efficiency of plagiarism detection, especially in terms of accurately identifying and locating plagiarized segments.

FAISS (Facebook AI Similarity Search) is an efficient similarity search and clustering tool that can quickly process large-scale text data and identify content highly similar to the query text. REPLUG (Retrieval-Enhanced Pre-trained Language Models), on the other hand, enhances the reasoning capability of language models by introducing a retrieval-augmented mechanism, enabling it to better understand and match semantically similar texts. In our research, we adjust the parameters of FAISS and REPLUG to optimize the model's search and matching strategies, allowing us not only to detect whether Student A has plagiarized from Student B, but also to precisely highlight the specific plagiarized parts, such as direct quotes, rephrasing, or simply changing certain names.

This improved approach, based on large-scale data and deep learning technology, can address the limitations of traditional plagiarism detection methods in terms of accuracy and efficiency, providing educational institutions with a more comprehensive and precise plagiarism detection tool. By combining automated model judgment with subsequent manual review, we can more efficiently address complex academic integrity issues and provide technical support to maintain fairness and integrity in the academic environment.